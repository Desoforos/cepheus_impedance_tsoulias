\documentclass[]{article}

\usepackage{amsthm} %qed
\usepackage[cmex10]{amsmath}
\usepackage{amsfonts}
\usepackage{amssymb}
\usepackage{graphicx}
\usepackage{hyperref}

\usepackage{mathtools}

\makeatletter
\DeclareRobustCommand\widecheck[1]{{\mathpalette\@widecheck{#1}}}
\def\@widecheck#1#2{%
    \setbox\z@\hbox{\m@th$#1#2$}%
    \setbox\tw@\hbox{\m@th$#1%
       \widehat{%
          \vrule\@width\z@\@height\ht\z@
          \vrule\@height\z@\@width\wd\z@}$}%
    \dp\tw@-\ht\z@
    \@tempdima\ht\z@ \advance\@tempdima2\ht\tw@ \divide\@tempdima\thr@@
    \setbox\tw@\hbox{%
       \raise\@tempdima\hbox{\scalebox{1}[-1]{\lower\@tempdima\box
\tw@}}}%
    {\ooalign{\box\tw@ \cr \box\z@}}}
\makeatother

\newcommand{\BIN}{\begin{bmatrix}}
\newcommand{\BOUT}{\end{bmatrix}}

\newcommand{\diff}[2]{\dfrac{\partial #1}{\partial #2}}
\newcommand{\norm}[1]{\left\| #1 \right\|}
\newcommand{\dt}{\Delta t}
\newcommand{\w}{\omega}
\newcommand{\dw}{\dot{\omega}}
\newcommand{\dv}{\dot{v}}
\newcommand{\comm}[2]{\left[#1,#2\right]}


\begin{document}

\title{\Large Integration of joint speed and acceleration in RBDyn}
\author{Adrien Escande}

\maketitle

\section{Motivation}
Given a joint with configuration $q$, velocity $\alpha$ and acceleration $\dot{\alpha}$, we are interested in computing (possibly approximately) the configuration of the joint after a duration $\dt$ for a constant acceleration (the case $\dot{\alpha} = 0$ corresponds to a constant velocity), that is integrate $\alpha(t)$ over the interval $\left[t,t+\dt\right]$. When (i) $q$ is living in an Euclidean space, and (ii) $q$, $\alpha$ and $\dot{\alpha}$ are expressed in the same frame, this is trivial:
\begin{equation}
	q(t+\dt) = q(t) + \alpha(t) \dt + \alpha \frac{\dt^2}{2}
\end{equation}
In RBdyn, this is the case for the revolute, prismatic and cylindrical joints, but not for the planar, spherical and free joints that respectively do not meet (ii), (i), (i) and (ii).
The goal of this note is to describe how to perform the integration in these more involved cases.

\section{Notations and assumptions}
Given $x \in \mathbb{R}^3$, $\hat{x}$ defines the skew-symmetric matrix corresponding to the cross product, \emph{i.e.} $\hat{x}u = x \times u$, or:
\begin{equation}
	\hat{x} = \BIN 0 & -x_3 & x_2 \\ x_3 & 0 & -x_1 \\ -x_2 & x_1 & 0 \BOUT
\end{equation}
Conversely, for a skew-symmetric matrix $S$, $ s = \widecheck{S} \in \mathbb{R}^3$ is the vector such that $\hat{s} = S$.

Without loss of generality, we integrate over $\left[0,\dt\right]$. For a function $x(t)$, we note $x_0 = x(0)$.

For two matrices $A$ and $B$, we have the matrix commutator $\comm{A}{B} = AB-BA$. $i$ is the identity matrix.


\section{Integration for a spherical joint}
A ball joint configuration is given by an element of SO(3). For the computations, we represent it as a rotation matrix $R$. \newline
In this section $\w$ denotes the velocity ($\w = \alpha$).

In RBDyn, $\w$ is expressed in the successor frame so that $\dot{R}(t) = R(t) \hat{\w}(t)$. Taking the transpose yields $\dot{R}^T = - \hat{\w} R^T$.

\subsection{Magnus expansion}
Given a linear ordinary differential equation $\dot{Y}(t) = A(t) Y(t)$, with initial value $Y(0) = Y_0$, an approximate solution can be obtained as
\begin{equation}
	Y(t) = \exp(\Omega(t))Y_0
\end{equation}
where $\Omega(t) = \Omega_1(t) + \Omega_2(t) + \ldots$ is the \emph{Magnus expansion} for the problem above.

The first terms are (\emph{cf} wikipedia), with the shortcut $A_i = A(t_i)$
\begin{align}
	\Omega_1(t)& = \int_0^{t}{A_1 dt_1}\\
	\Omega_2(t)& = \int_0^{t}{\int_0^{t_1}{\comm{A_1}{A_2} dt_2}\ dt_1}\\
	\Omega_3(t)& = \int_0^{t}{\hspace{-6pt}\int_0^{t_1}{\hspace{-6pt}\int_0^{t_2}{\comm{A_1}{\comm{A_2}{A_3}} + \comm{A_3}{\comm{A_2}{A_1}} dt_3}\ dt_2}\ dt_1}\\
	\Omega_4(t)& = \int_0^{t}\hspace{-6pt}\int_0^{t_1}\hspace{-6pt}\int_0^{t_2}\hspace{-6pt}\int_0^{t_3}
		\comm{\comm{\comm{A_1}{A_2}}{A_3}}{A_4} + \comm{A_1}{\comm{\comm{A_2}{A_3}}{A_4}} \\
		& + \comm{A_1}{\comm{A_2}{\comm{A_3}{A_4}}} + \comm{A_2}{\comm{A_3}{\comm{A_4}{A_1}}}
	dt_4\ dt_3\ dt_2\ dt_1 \nonumber
\end{align}

\subsection{Application to $\dot{R}^T = - \hat{\w} R^T$}
For the problem $\dot{R}^T(t) = - \hat{\w}(t) R^T(t)$, $R(0) = R_0$, we have that $R^T(t) = \exp(\Omega(t)) R_0^T(t)$. Thus $R(t) = R_0 \exp(\Omega(t))^T = R_0  \exp(-\Omega(t))$.

With $\w(t) = \w_0 + t \dot{\w}_0$,
\begin{align}
	\Omega_1(t) & = -(\widehat{\w_0} + \frac{t}{2}\widehat{\dw_0})t \\
	\Omega_2(t) & = \frac{1}{12} \widehat{\dw_0 \times \w_0} t^3 \\
	\Omega_3(t) & = -\frac{1}{240} \widehat{\dw_0 \times (\dw_0 \times \w_0)}t^5 \\
	\Omega_4(t) & = \frac{1}{5040} (\norm{\dw_0}^2 t^2 + 7 \dw_0^T\w_0 t + 7 \norm{\w}^2) \widehat{\dw_0 \times \w_0}t^5 \\
	\Omega_5(t) & = \frac{1}{241920} \left((5\norm{\dw_0}^2 t^2 + 24\dw_0^T\w_0 t + 24\norm{\w}^2) \widehat{\dw_0 \times (\w_0 \times \dw_0)}\right. \\
	& \left.+ 8\norm{\w_0 \times \dw_0}^2 (\widehat{\w_0} + \frac{t}{2}\widehat{\dw_0})\right) t^7
\end{align}
(adapted from \footnote{\scriptsize \url{https://cwzx.wordpress.com/2013/12/16/numerical-integration-for-rotational-dynamics/}} 3 and checked in Matlab for the three first ones, deduced from Matlab computations for the 4th and 5th ones).

The first two terms are usually enough. The other ones give an idea of the error magnitude and can be computed if needed. Note that when the velocity is constant ($\dw = 0$), $-\Omega(t)$ reduces to $t \widehat{\w_0}$, so that $R(t) = R_0 \exp(\widehat{t\w_0})$, which is the expected result.

Since the squared norm of $\Omega_1$ is $\norm{\dw_0}^2 \frac{t^2}{4} + \dw_0^T \dw t + \norm{\w_0}^2$, we can bound quite tightly $\left|\norm{\dw_0}^2 t^2 + 7 \dw_0^T\w_0 t + 7 \norm{\w}^2 \right|$ and $\left|5\norm{\dw_0}^2 t^2 + 24\dw_0^T\w_0 t + 24\norm{\w}^2\right|$ by $7\norm{\Omega_1}^2$ and $24\norm{\Omega_1}^2$, respectively. Consequently,
\begin{align}
	\norm{\Omega_3} &= \frac{\norm{\dw_0} t^2}{20} \norm{\Omega_2} &&= \frac{r \norm{\dw_0} t^2}{20} \norm{\Omega_1}\\
	\norm{\Omega_4} &\leq \frac{1}{60} \norm{\Omega_1}^2\norm{\Omega_2} &&= \frac{r}{60} \norm{\Omega_1}^3\\
	\norm{\Omega_5} &\leq \frac{\norm{\dw_0} t^2}{840} \norm{\Omega_1}^2 \norm{\Omega_2} + \frac{1}{210} \norm{\Omega_1} \norm{\Omega_2}^2 &&= \frac{r \norm{\dw_0} t^2 + 4r^2}{840} \norm{\Omega_1}^3
\end{align}
where $r$ is the ratio between $\norm{\Omega_1}$ and $\norm{\Omega_2}$: $\norm{\Omega_2} = r \norm{\Omega_1}$.


\subsection{Implementation details}
Since we want to compute $\exp(-\Omega(t))$ and that $\Omega$ is skew-symmetric (whatever the number of terms used), \emph{magnusExpansion} returns -$\widecheck{\Omega}$.
\newline
All norms are computed in squared form to avoid taking square roots as much as possible. Noting $\mu = \norm{\dw_0} t^2$,
\begin{align}
\norm{\Omega_3}^2 &= \frac{\mu^2}{400} \norm{\Omega_2}^2\\
\norm{\Omega_4}^2 &\leq \frac{1}{3600} \norm{\Omega_1}^4\norm{\Omega_2}^2\\
\norm{\Omega_5}^2 &\leq \frac{\norm{\Omega_1}^2 \norm{\Omega_2}^2}{840^2} (\mu \norm{\Omega_1} + 4 \norm{\Omega_2})^2\\ &= \frac{\norm{\Omega_1}^2 \norm{\Omega_2}^2}{840^2}(\mu^2 \norm{\Omega_1}^2 + 8\sqrt{\mu^2 \norm{\Omega_1}^2 \norm{\Omega_2}^2} + 16 \norm{\Omega_2}^2)
\end{align}
where the last line is formed to use a single squared root.

\section{Integration for a free joint}
In this section $v$ and $\w$ denote the linear and angular velocity, respectively ($\alpha = \BIN \w \\ v \BOUT$).

Free joints are peculiar beasts in RBDyn (following Featherstone's book). Their configuration is an element of SE(3) represented by a translation $x$ expressed in parent frame, followed by a rotation $R$. However their velocity is represented in successor frame. Consequently, the linear velocity of a joint in parent frame is given by $\dot{x}(t) = R(t)v(t)$. This makes the integration complicated when $R$ is not constant. When $R$ is constant, we simply have (for constant acceleration)
\begin{equation}
	x(t) = x_0 + R(v_0t + \dv_0 \frac{t^2}{2})
\end{equation}

\subsection{Constant velocity}
We suppose that $v(t) = v_0$ and $\w(t) = \w_0$, so that $\dot{x}(t) = R_0\exp(t \widehat{\w_0}) v_0$. Using Rodrigues' formula, we have that
\begin{equation}
	\dot{x}(t) = R_0 \left(I + \sin(t \norm{\w_0})\frac{\widehat{\w_0}}{\norm{\w_0}} + \left(1-\cos(t \norm{\w_0})\right) \frac{\widehat{\w_0}^2}{\norm{\w_0}^2} \right) v_0
\end{equation}
Taking the integral between $0$ and $t$ yields
\begin{equation}
	x(t) = x_0 + R_0 \left(tI + (1-\cos(t \norm{\w_0}))\frac{\widehat{\w_0}}{\norm{\w_0}^2} + \left(t-\frac{1}{\norm{\w_0}}\sin(t \norm{\w_0})\right) \frac{\widehat{\w_0}^2}{\norm{\w_0}^2} \right) v_0
\end{equation}

\subsection{Constant acceleration}
In the case of constant acceleration, we can't rely on analytical integration anymore because of the rotation part. We then have to use numerical integration. We use Simpson's rule: let $h = \frac{\dt}{2n}$, $t_i = ih,\ i=0..2n$ and $f_i = f(t_i)$, then
\begin{align}
	\int_0^{\dt} f(t) dt &= \frac{h}{3}\left(f_0 + 4 f_1 + 2f_2 + 4 f_3 + 2 f_4 + \ldots + 4f_{2n-1} + f_{2n}\right) + \epsilon_n \\
	& = \frac{h}{3} \left(f_0 + \sum_{j=0}^{n-1} (4 f_{2j+1} + 2 f_{2j+2}) + 4f_{2n-1} + f_{2n}\right) + \epsilon_n
\end{align}
with
\begin{equation}
	\epsilon_n = \frac{nh^5}{90}f^{(4)}(\xi) = \frac{\dt^5}{2880n^4}f^{(4)}(\xi)
\end{equation}
for some $\xi \in \left[0,\dt\right]$.
\newline
For $f(t) = R(t)v(t)$ with $v(t) = v_0 + t \dv_0$, $\w(t) = \w_0 + t \dw_0$ and $\dot{R}(t) = R(t)\hat{\w}(t)$, we have, dropping the dependence in $t$:
\begin{align}
	&f^{(4)} = R \left((\norm{\w}^4-3\norm{\dw_0}^2)v -12\dw_0^T \w \dv_0 + (4\dw_0^T \dv_0 - \norm{\w}^2\w^T v)\w\right. \nonumber\\
	& \left. + (3\dw_0^T v + 8\w^T \dv_0)\dw_0+(5\dw_0^T \w v + 4\norm{\w}^2 \dv_0) \times \w + (2 \w^T v\w + \norm{\w}^2 v) \times \dw_0\right)\nonumber
\end{align}
(see Appendix for details)
\newline
Let's note $M$ the maximum norm of $f^{(4)}$ over $\left[0,\dt\right]$. Then $\norm{\epsilon_n} \leq \frac{\dt^5}{2880n^4}M$, so that a sufficient condition for $\norm{\epsilon_n}$ to be smaller than a target precision $p$ is
\begin{equation}
  n \geq \frac{\dt}{2} \sqrt[4]{\frac{M\dt}{180p}}
\end{equation}

Computing $M$ would be too costly, but a good approximation is to take $M = \max(f^{(4)}(0),f^{(4)}(\dt))$: empirically, it is generally true and when it's not the error is usually small or doesn't change the order of magnitude\footnote{We tested for 1 millions random values of $(R,v,\w,\dv,\dw)$ obtained with Matlab's \emph{randn}. For $\dt=0.1$, the max is not at the bounds in about $0.03\%$ of the cases, and the error is less than $4\%$. For $\dt=1$, the max is not at the bounds in about $0.25\%$ of the cases, and the error is usually less than $10\%$, but go up to $330\%$ in one case. For $\dt=5$, the max is not at the bounds in about $0.1\%$ of the cases, and the error is usually less than $20\%$, but go up to $500\%$ in a few cases.}.


\subsection{Implementation details}
An important remark is that $\norm{f^{(4)}}$ is independent of $R(t)$ as $R(t)$ is orthogonal. This means it can be computed without having to integrate the rotation part of the free joint.
\newline



\section{Planar joint}
In the current version of RBDyn, all translation quantities are expressed in the successor's frame. Noting $x$ the translation part for this joint and $\theta$ the rotation angle, we have
\begin{align}
&\dot{x}(t) = A(t) x(t) + R(\theta(t)) v(t)\\
&\mbox{with}\ A(t) = \BIN 0 & \w(t) \\ -\w(t) & 0 \BOUT \ \mbox{and}\ R(u) = \BIN \cos{u} & -\sin{u} \\ \sin{u} & \cos{u}\BOUT \nonumber
\end{align}
We define
\begin{equation}
	B(t) = \BIN 0 & \theta(t) \\ -\theta(t) & 0 \BOUT
\end{equation}
so that $\dot{B} = A$ and $e^B = R(-\theta) = R^T(\theta)$. $A$ and $B$ commute, so that the solution to the above differential equation for $x(0) = x_0$ and $\theta(0) = \theta_0$ is
\begin{equation}
	x = R^T(\theta)\left(R_0 x_0 + \int_0^t R(\theta(u))v(u) du \right) \label{eq:planarSol}
\end{equation}


\subsection{Constant velocity}
We suppose here that $v(t) = v_0$, $\w(t) = \w_0$ and note $\theta_0 = \theta(0)$.\newline
In this case, the integration of $R(\theta)v$ has an analytical form:
\begin{equation}
	\int R(\theta(t)) dt = \int R(\theta_0+ t \w) dt = \frac{1}{\w} \BIN \sin{\theta(t)} & \cos{\theta(t)} \\ -\cos{\theta(t)} & \sin{\theta(t)} \BOUT
\end{equation}
so that
\begin{align}
	x & = R(t\w)^T x_0 + \frac{1}{\w}\left( \BIN \sin{t \w} &-\cos{t \w} \\ \cos{t \w} & \sin{t \w} \BOUT + \BIN 0 & 1 \\ -1 & 0 \BOUT \right) v \\
	  & = R(t\w)^T x_0 + t \BIN \frac{\sin{t \w}}{t \w} & \frac{1-\cos{t \w}}{t \w} \\ \frac{\cos{t \w}-1}{t \w} & \frac{\sin{t \w}}{t \w} \BOUT v
\end{align}

\subsection{Constant acceleration}
As for free joints, in the case of constant acceleration, analytical formula is not an option. We use Simpson's rule to integrate $f(t) = R(\theta(t))v(t)$ and use eq.~(\ref{eq:planarSol}) to get the solution.
The error of the integration is based on
\begin{equation}
 f^{(4)} = \left(\w^4 - 3 \dw_0^2 \right) R v - 6 \dw_0 \w^2 U v -12 \dw_0 \w R \dv_0  - 4 \w^3 U \dv_0
\end{equation}
where $R$ stands for $R(\theta(t))$ and
\begin{equation}
	U = \BIN -\sin{\theta(t)} & -\cos{\theta(t)} \\ \cos{\theta(t)} & -\sin{\theta(t)} \BOUT
\end{equation}
The $4$-th derivative is obtained by direct computation, noting that $\dot{R} = \w U$ and $\dot{U} = -\w R$.

Defining $M = R^T U = \BIN 0 & -1 \\ 1 & 0 \BOUT$ and noting that $R^T R = I$, we have that
\begin{equation}
	\norm{f^{(4)}} = \norm{R^T f^{(4)}} = \norm{\left(\w^4 - 3 \dw_0^2 \right) v - 6 \dw_0 \w^2 M v -12 \dw_0 \w \dv_0  - 4 \w^3 M \dv_0}
\end{equation}
wich is independent of $\theta$.


\appendix
\section{Computation of $f^{(4)}$ in the free joint case}
Let $f(t) = R(t)v(t)$ with $v(t) = v_0 + t \dv_0$, $\w(t) = \w_0 + t \dw_0$ and $\dot{R}(t) = R(t)\hat{\w}(t)$. We have
\begin{align*}
	\dot{f} &= \dot{R}v + R\dot{v}\\
	&= R \left(\hat{\w}v+\dv_0 \right)\\
	\ddot{f} &= R \left( \hat{\w}^2v + 2\hat{\w}\dv_0 + \widehat{\dw_0} v\right)\\
	f^{(3)} &= R \left(\hat{\w}^3v + 3\hat{\w}^2\dv_0 + 2\hat{\w} \widehat{\dw_0} v + \widehat{\dw_0} \hat{\w} v + 3 \widehat{\dw_0} \dv_0 \right)\\
	f^{(4)} &= R \left(\left(\hat{\w}^4 + 3\hat{\w}^2 \widehat{\dw_0} + \widehat{\dw_0} \hat{\w}^2 + 2\hat{\w}\widehat{\dw_0}\hat{\w} + 3\widehat{\dw_0}^2\right)v + 4\left(\widehat{\dw_0}\hat{\w} + 2 \hat{\w}\widehat{\dw_0} + \hat{\w}^3 \right) \dv_0 \right)
\end{align*}

Using the fact that $\hat{x}\hat{y} = y x^T - x^T y I$, we have the following simplifications:
\begin{align*}
	\hat{\w}^4 &= \norm{\w}^4 I - \norm{\w}^2 \w\w^T \\
	\hat{\w}^3 &= -\norm{\w}^2 \hat{w} \\
	\widehat{\dw_0}^2 &= \dw_0 \dw0^T - \norm{\dw_0}^2 I\\
	3\hat{\w}\widehat{\dw_0} + \widehat{\dw_0}\hat{\w}^2 + 2\hat{\w}\widehat{\dw_0}\hat{\w} &= 2 (\w \times \dw_0) \w^T - \norm{\w}^2\widehat{\dw_0} - 5\dw_0^T\w \hat{\w}\\
	\widehat{\dw_0}\hat{\w} + 2\hat{\w}\widehat{\dw_0} + \hat{\w}^3 &= \w \dw_0^T + 2 \dw_0 \w^T - 3\dw_0^T\w I -\norm{\w}^2\hat{\w}
\end{align*}
which yield the result.

\end{document}
